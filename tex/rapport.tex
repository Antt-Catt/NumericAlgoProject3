\documentclass{article}

\usepackage[french]{babel}
\usepackage[utf8]{inputenc}
\usepackage{graphicx}
\usepackage{subcaption}
\usepackage{hyperref}

%%%%%%%%%%%%%%%% Lengths %%%%%%%%%%%%%%%%
\setlength{\textwidth}{15.5cm}
\setlength{\evensidemargin}{0.5cm}
\setlength{\oddsidemargin}{0.5cm}

%%%%%%%%%%%%%%%% Variables %%%%%%%%%%%%%%%%
\def\projet{3}
\def\titre{Compression d'image à travers la factorisation SVD}
\def\groupe{4}
\def\equipe{6}
\def\responsible{jeamartinez}
\def\secretary{acattarin}
\def\others{mchellaf, siducamp}

\begin{document}

%%%%%%%%%%%%%%%% Header %%%%%%%%%%%%%%%%
\noindent\begin{minipage}{0.98\textwidth}
  \vskip 0mm
  \noindent
  { \begin{tabular}{p{7.5cm}}
      {\bfseries \sffamily
        Projet \projet} \\ 
      {\itshape \titre}
    \end{tabular}}
  \hfill 
  \fbox{\begin{tabular}{l}
      {~\hfill \bfseries \sffamily Groupe \groupe\ - Equipe \equipe
        \hfill~} \\[2mm] 
      Responsable : \responsible \\
      Secrétaire : \secretary \\
      Codeurs : \others
    \end{tabular}}
  \vskip 4mm ~

  ~~~\parbox{0.95\textwidth}{\small \textit{Résumé~:Le but de ce projet 
  consiste à programmer un algorithme permettant de faire de la compression 
  d'images en utilisant des techniques matricielles basée sur la factorisation SVD.
  Ce type d'algorithme est à relier aux algorithmes de compression avec pertes, 
  dont le plus connu est certainement l'algorithme de compression JPEG, 
  lui-même basé usuellement sur la Discrete Cosine Transform (DCT), une transformation voisine de la transformée de Fourier discrète.} \sffamily }
  \vskip 1mm ~
\end{minipage}

%%%%%%%%%%%%%%%% Main part %%%%%%%%%%%%%%%%
\section{Transformation de Householder}
\label{sec:transfo_householder}

\subsection{Construction d'une matrice de Householder}
\label{ssec:construc_householder}

\subsection{Application d'une matrice de Householder}
\label{ssec:appli_householder}

\subsection{Etude de compléxité}
\label{ssec:complex_householder}


\section{Mise sous forme bidiagonale}
\label{sec:forme_bidiag_}

\subsection{Implémentation}
\label{ssec:implem_bidiag_}

\subsection{Analyse}
\label{ssec:analyse_bidiag}


\section{Transformations QR}
\label{sec:transfo_qr}

\subsection{Implémentation}
\label{ssec:implem_qr}

\subsection{Etude de convergence}
\label{ssec:conv_qr}

\subsection{Analyse}
\label{ssec:analyse_qr}

La matrice $S$ est bidiagonale dès le début de la fonction.

Pour chaque itération de la boucle \verb|for| de la fonction \verb|transfo_qr| :
\smallskip

\noindent La matrice $R1$ est obtenue grâce à une décomposition QR de $S$.

\noindent La matrice $R2$ est obtenue grâce à une décomposition QR de $R2$.

\noindent La matrice $S$ prend les valeurs de $R2$.

\smallskip
Or, la matrice obtenue avec une décomposition QR d'une matrice bidiagonale est également bidiagonale. %à prouver ?

A la première itération, $R1$ est donc bidiagonale (issue de la décomposition QR de $S$). De même, $R2$ est également bidiagonale. Finalement, $S$ prend les valeurs de $R2$. L'invariant est donc vrai après la première itération.

\smallskip

Nous prouvons l'induction exactement de la même manière, puisque nous supposons que $S$ est bidiagonale au début de l'itération.

\subsection{Optimisation pour une matrice bidiagonale}
\label{ssec:opti_bidiag_qr}

\subsection{Mise en forme}
\label{ssec:mise_en_forme_qr}


\section{Application à la compression d'image}
\label{sec:appli_compr_img}

\subsection{Compression}
\label{ssec:compr_img}

\subsection{Analyse quantitative}
\label{ssec:quanti_img}

\subsection{Analyse qualitative}
\label{ssec:quali_img}

\end{document}
